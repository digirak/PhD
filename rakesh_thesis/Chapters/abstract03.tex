\chapter*{abstract}
Extracted spectra are an increasing part of exoplanet science, particularly with the advent of the James Webb Space telescope. 
The scientific goals of such spectra range from molecular characterization, molecular species detection to constraining abundances in exoplanet data.
In direct imaging the availability of spectra in high contrast images has allowed astronomers to detect exoplanets and characterize them.
With increasing data sizes and the availability of high resolution direct imaging spectra from high resolution spectrographs, there are both computational and accuracy limitations brought by the traditional techniques used to analyze these spectra.
The advent of advanced data science algorithms particularly machine learning algorithms have been suggested as an alternative to traditional spectral processing.
This part is dedicated to studying the effect of ML algorithms when used directly on spectra extracted from direct imaging data.
To that end we have two introductory chapters;
in the first one  we will first introduce data processing methods for direct imaging spectra and in the second we will then introduce ML algorithms and specifically the classes of algorithms used in this chapter.
This is followed by the data chapter in which we will define the science objectives and the data that is used to meet these science objectives.
This is followed by two methods chapters; 
the first of which describes the cross correlation based algorithm and the results it produces on the benchmarking scale and its scientific impact,
The second methods chapter describes the ML algorithms and their results
The final chapter in this part will discuss the results and derive a conclusion from both the experiments and finally define what next step this piece of research has motivated us to do.
