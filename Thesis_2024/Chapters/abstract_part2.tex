\chapter*{Part summary}
Exoplanet characterization using emission/absorption spectra of exoplanets is becoming an increasing part of exoplanet science particularly\@ 
with the advent of space based telescopes such as the James Webb space telescope and future ground based telescopes such as ELT.\@
%spectra are an increasing part of exoplanet science, particularly with the advent of the James Webb Space telescope. 
The scientific goals of such spectra range from detection of molecular species in  exoplanets and constraining abundances of specific\@
molecular species in detected exoplanets.
%In direct imaging the availability of spectra in high contrast images has allowed astronomers to detect exoplanets and characterize them.
Typically, the methods used to detect molecules or contrain their abundances, require comparing spectra derived from direct imaging instruments and \@ 
comparing them to known template spectra and interpret the similarities between the two.
The goals of such comparison has been produce a characterization outcome such as presence of specific molecules on confirmed exoplanets and hence\@
these sort of outcomes are an after thought of detecting exoplanets, even though the molecular absorption features that lend to this sort of characterization \@
are very specific to exoplanets.

Comparison of spectra of exoplanets with this template is usually conducted using cross correlation analysis. The cross correlations are a well known method \@
to compare spectra and express their similarity in the cross correlated values. 
Typically, a high cross correlation corresponds to a high similarity between the spectra.
In this chapter, we provide a framework where cross correlations can be used to perform simultaneous detection and characterization, we lay down the baseline\@
of the following,
\begin{itemize}
    \item generating a large number of simulated exoplanet spectra that have the following variable parameters contrast ($C$), spectral resolution($R$) \@
    and signal to noise ratio ($\rm{SNR}$).
    \item quantify the performance of a cross correlation based algorithm as exoplanet detection tool, the detection and characterization sensitivity\@
    of such a tool.
    \item a baseline performance measure which places the limitations of using spectra to perform simultaneous detection and characterization.
\end{itemize}
This chapter thus establishes the utility of a cross correelation based algorithm and the methodology to evaluate\@
it.

To this end, we start with introducing spectral data processing to detect and characterize exoplanets in \S\ref{chap: II.1}.
This is then followed by an introduction to cross correlation and our implementation of it. 
%\comment{This is to be merged or not}.
This makes up the `introductory' portion of this part. 
The `methods' of this part is related to generating data which covers the full spread of $\left(C, R, \rm{SNR}\right)$\@
that allows us to generate data which is astronomically relevant and is thus a good test bed for evaluating our \@
cross correlation based algorithm.
In this chapter we also describe our evaluation criteria for both detection and characterization.