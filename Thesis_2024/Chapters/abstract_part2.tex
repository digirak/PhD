\chapter*{Thesis summary}
%broad intro
High contrast imaging (HCI) is a field of astrophysics that deals with obtaining of images that pixels that have extremely bright sources (stars for example) and extremely faint sources (exoplanets for example).
This technique has been used effectively to detect exoplanets with wide orbits which are not easily detectable with indirect detection techniques.
My thesis primarily focuses on data obtained from direct imaging instruments and therefore a broad introduction the field of exoplanet detection and characterization\@
using high contrast imaging.
This forms the first chapter of the introduction.
HCI data has become increasingly complex and with modern imaging techniques it is possible to have a large dataset with several dimensions.
This automatically leads to introducing a modern data analysis technique, machine learning algorithms.
Thus the next chapter will focus on introducing machine learning and its application to astronomical datasets.
%%Explain detection and characterization and the need for ML
Detection involves analyzing the images from the direct imaging data and using specific algortithmic strategies to identify those pixels that contain the exoplanet.
Whereas characterization of this detection requires astronomers to analyze spectra from the detection location.
Several imaging strategies such as \citep[ADI, ][]{2006MaroisADI} have been used to detect exoplanets, more details on this follows in 
\S\ref{chap: I.1}.
Naturally, this implies that characterization comes as an afterthought of detection.
To reduce this gap, a method was suggested to combine spectral and spatial data, in essence combining the detection 
and characterization step.
Such an example is \citet[][]{2002SparksSDI}.
This step allows us to look at spectra both in combination with images and independently as spectra alone.
This has been made possible with the presence of Integral Field Spectrographs (IFS) and the next generation of adapative optics based 
telescopes.
When this step is combined with images we produce what are known as molecular maps an idea that is explained in 
\S\ref{chap: I.1} and our adaptions of this concept is desribed later in this thesis.
A brief description of spectral data processing is thus given to conclude this introductory that explains the scientific relevance of spectra data processing.
ML algorithms have been recently used quite effectively in exoplanet detection using high contrast images through SODINN, \citet[][]{2018Gomez}
and NA-SODINN, \citet[][]{2023Carlito}.
These algorithms have been used only on the HCI data without leveraging the spectral or temporal dimensions. 
The success of these algorithms and the presence of spectral pre-processing algorithms such as molecular mapping, \citet[][]{2018AHoeijmakersMM},
leaves the question of whether it would be possible to combine spectral, spatial and temporal data to improve detection and characterization of exoplanets.

This thesis is aimed at exploring methods of combining spectral, spatial and temporal data with ML algorithms and establishing the sensitivity 
of such combinations. 
The introductory part establishes the state of the art in terms of both science and the instruments and the post processing algorithms that produce the scienctific results.
Methods such as a molecular mapping exploit intrinsic features in the spectra, combine them with high contrast images to produce detection maps of specific molecules.
There have been several algorithms \citep[e.g HRSDI][]{2019Haffert} that perform spectral pre-processing to then be combined with techniques such as molecular mapping.
All of these algorithms use cross correlation as their principal analysis methodology. 
While cross correlations are well studied in signal processing methodologies, more about this in the Introduction to Cross correlations chapter in Part II,
they have still not been used to benchmark detectability of exoplanets in high contrast imaging datasets.
This is the gap that will be filled by my study of cross correlations to detect and characterize exoplanet spectra.
In the second chapter of this Part we will described our method of generating large number of synthetic spectra and their analyses framework for both detection and characterization.
In the third chapter of this Part we will describe some of the tests we conducted and the conclusion on the effectiveness of detection and characterization using spectra alone.
We will finally conclude this part with the learning that while it is possible to detect exoplanets at high contrast (and even characterize them) it requires spectra to have unreasonably high signal to noise values.
This Part will motivate thus, whether it is possible to use the image dimension to get more sensitive detection and/or characterization.

This thesis focuses on high contrast imaging data, and therefore it stands to reason that high contrast images themselves are tantamount to our algorithms.
While Part II focused on 1D spectra to establish the baseline of what can be achieved with cross correlation, Part III will describe our experiments with using imaging part to attempt more sensitive detections.
The introduction of this Part is already established in Part I which introduces high contrast images, therefore the first chapter will begin with adapting our techniques to use with HCI.
This chapter will describe our successful re-detection of HD$142527$b and our failure to detect PDS$70$b. 
While there has been a phsyically motivated explanation by \cite{2023Cugno}, we conclude that the reason we don't detect PDS$~70$b is that the measured signal to noise on this exoplanet for our data is just out of reach of our processing.
We also conclude this Chapter with the understanding that the image dimension itself does not lend to improvement in detection sensitivity, and in fact our detection benchmark itself is a bit polluted with the use of the imaging dimension.
In Chapter$~8$ (second chapter of this Part) we will describe our efforts at using the temporal dimension with cross correlation maps. 
Finally, we will establish that our sensitivity analysis is better served with the use of RoC curves with images than using contrast curves.
The last chapter in this part is a discussion chapter on the `classical' post processing algorithms that this thesis has introduced to the field of exoplanet detection and characterization for HCI spectra.
We will end this Part and Chapter with the need to use modern ML algorithms with such data and need to study their use.

Part IV will focus entirely on the application of ML algorithms to HCI data.
We will begin with describing our efforts to apply ML algorithms directly to spectra. 
This will comprise the entire Chapter$~10$, where we apply ML algorithms to the data generated in Chapter$~5$. 
We will describe the results and the limitations of ML algrorithms when used directly with spectra and motivate the use of images with ML algoriothms to end this chapter.
Chapter$~11$ will describe ML algorithms that were developed to work with 2D and 3D data.
We will end this chapter with describing the method this thesis uses to interpret the result of the detection maps produced by ML algorithms.
Chapter$~12$ will describe the training and testing algorithms developed for the algorithms in Chapter$~11$ and its results. 
Finally, we will conclude this Part with a discussion on the use of ML algorithms to detect exoplanets in HCI data with spectra.
Each of these parts correspond to several projects that we completed as part of my thesis that explores detection and characterization of exoplanets.
This however, motivates several discussion points which are broader in scope.

Part V, thus focuses on discussing several aspects of my thesis. 
To begin with we discuss the relevance of this study to the exoplanet community in Chapter$~14$.
We then discuss the limitations ML algorithms and why standard cross correlation analysis is more suited to directly apply to spectra in Chapter$~15$.
We conclude this chapter with a broad discussion on the use of multi-dimensional data which best leverages ML algorithms.
The final Chapter of this thesis concludes with some ways forward where the analysis we have performed as part of this thesis can be best leveraged by the scientific commununity.  

%Part II focuses on describing the development of a cross correlation algorithm that processes spectra to simulataneously\@
%detect and characterize exoplanets.
%In this part we have $3$ chapters starting with the introduction to spectral data processing, followed by an introduction
%to cross correlation analysis. 
%In this chapter we also describe our implementation of cross correlation with the pre-processing.
%This is followed by a description of our testbench which is composed of data generation, testing and evaluation.
%Finally, we end this part with a retrospective on the limitations on spectral data processing. 

In Part III we discuss the use of our cross correlation algorithm with medium resolution spectra and high contrast images.
This part contains $4$ chapters starting an introduction to high contrast imaging datasets and their dimensionalities.
this is followed by our algorithm to use these dimensions to perform simulataneous detections and characterizations of exoplanets.
