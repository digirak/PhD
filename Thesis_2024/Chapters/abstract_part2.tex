\chapter*{chapter summary}
Exoplanet characterization using emission/absorption spectra of exoplanets is becoming an increasing part of exoplanet science particularly\@ 
with the advent of space based telescopes such as the James Webb space telescope and future ground based telescopes such as ELT.\@
%spectra are an increasing part of exoplanet science, particularly with the advent of the James Webb Space telescope. 
The scientific goals of such spectra range from detection of molecular species in  exoplanets and constraining abundances of specific\@
molecular species in detected exoplanets.
%In direct imaging the availability of spectra in high contrast images has allowed astronomers to detect exoplanets and characterize them.
Typically, the methods used to detect molecules or contrain their abundances, require comparing spectra derived from direct imaging instruments and \@ 
comparing them to known template spectra and interpret the similarities between the two.
The goals of such comparison has been produce a characterization outcome such as presence of specific molecules on confirmed exoplanets and hence\@
these sort of outcomes are an after thought of detecting exoplanets, even though the molecular absorption features that lend to this sort of characterization \@
are very specific to exoplanets.

Comparison of spectra of exoplanets with this template is usually conducted using cross correlation analysis. The cross correlations are a well known method \@
to compare spectra and express their similarity in the cross correlated values. 
Typically, a high cross correlation corresponds to a high similarity between the spectra.
In this chapter, we provide a framework where cross correlations can be used to perform simultaneous detection and characterization, we lay down the baseline\@
of the following,
\begin{itemize}
    \item generating a large number of simulated exoplanet spectra that have the following variable parameters contrast ($C$), spectral resolution($R$) \@
    and signal to noise ratio ($\rm{SNR}$).
    \item quantify the performance of a cross correlation based algorithm as exoplanet detection tool, the detection and characterization sensitivity\@
    of such a tool.
    \item a baseline performance measure which places the limitations of using spectra to perform simultaneous detection and characterization.
\end{itemize}
This chapter thus establishes the utility of a cross correelation based algorithm and the methodology to evaluate\@
it.
To that end we have two introductory chapters;
in the first one  we will first introduce data processing methods for direct imaging spectra and in the second we will then introduce ML algorithms and specifically the classes of algorithms used in this chapter.
This is followed by the data chapter in which we will define the science objectives and the data that is used to meet these science objectives.
This is followed by two methods chapters; 
the first of which describes the cross correlation based algorithm and the results it produces on the benchmarking scale and its scientific impact,
The second methods chapter describes the ML algorithms and their results
The final chapter in this part will discuss the results and derive a conclusion from both the experiments and finally define what next step this piece of research has motivated us to do.
