\chapter*{Thesis summary}
%broad intro
High contrast imaging (HCI) is a field of astrophysics that deals with obtaining of images that pixels that have extremely bright sources (stars for example) and extremely faint sources (exoplanets for example).
This technique has been used effectively to detect exoplanets with wide orbits which are not easily detectable with indirect detection techniques.
My thesis primarily focuses on data obtained from direct imaging instruments and therefore a broad introduction the field of exoplanet detection and characterization\@
using high contrast imaging.
This forms the first chapter of the introduction.
HCI data has become increasingly complex and with modern imaging techniques it is possible to have a large dataset with several dimensions.
This automatically leads to introducing a modern data analysis technique, machine learning algorithms.
Thus the next chapter will focus on introducing machine learning and its application to astronomical datasets.
%%Explain detection and characterization and the need for ML
Detection involves analyzing the images from the direct imaging data and using specific algortithmic strategies to identify those pixels that contain the exoplanet.
Whereas characterization of this detection requires astronomers to analyze spectra from the detection location.
Several imaging strategies such as \citep[ADI, ][]{2006MaroisADI} have been used to detect exoplanets, more details on this follows in 
\S\ref{chap: I.1}.
Naturally, this implies that characterization comes as an afterthought of detection.
To reduce this gap, a method was suggested to combine spectral and spatial data, in essence combining the detection 
and characterization step.
Such an example is \citet[][]{2002SparksSDI}.
This step allows us to look at spectra both in combination with images and independently as spectra alone.
This has been made possible with the presence of Integral Field Spectrographs (IFS) and the next generation of adapative optics based 
telescopes.
When this step is combined with images we produce what are known as molecular maps an idea that is explained in 
\S\ref{chap: I.1} and our adaptions of this concept is desribed later in this thesis.
A brief description of spectral data processing is thus given to conclude this introductory that explains the scientific relevance of spectra data processing.
ML algorithms have been recently used quite effectively in exoplanet detection using high contrast images through SODINN, \citet[][]{2018Gomez}
and NA-SODINN, \citet[][]{2023Carlito}.
These algorithms have been used only on the HCI data without leveraging the spectral or temporal dimensions. 
The success of these algorithms and the presence of spectral pre-processing algorithms such as molecular mapping, \citet[][]{2018AHoeijmakersMM},
leaves the question of whether it would be possible to combine spectral, spatial and temporal data to improve detection and characterization of exoplanets.

This thesis is aimed at exploring methods of combining spectral, spatial and temporal data with ML algorithms and establishing the sensitivity 
of such combinations. 
The introductory part establishes the state of the art in terms of both science and the instruments and the post processing algorithms that produce the scienctific results.
%The introductory part is followed by the first original part of this thesis called the cross correlation of exoplanet spectra from high contrast imaging data for simulataneous detection and characterization.
%The theme of this part is to explore the data processing necessary to use stand-alone spectra to detect and characterize exoplanets.
%This part begins with an introduction to cross correlation as a data processing technique \S \ref{chap:II.1}. 
%In this chapter we will describe the broad definition of cross correlation and end with how we have adapted the cross correlation to  work with exoplanet spectra and the necessary pre-processing needed for this.
%Explain contribution here
%The broad aim of my thesis is to combine high contrast imaging data with spectroscopic observations and thus create a unified framework to simulataneously,
%detect and characterize an exoplanet candidate.
%The goal of this framework would be to evaluate if a candidate exoplanet can indeed be verified as an exoplanet.
%In order to do this, it is important to test the framework both from the point of view of statistics and astrophysics.
%Hence, we present in this thesis three frameworks which work on data with different dimensionalities.
%Each framework is studied for both its strengths and limitations, and in this regard, my thesis is thus divided into three parts that focus on each of these frameworks.

Part II focuses on describing the development of a cross correlation algorithm that processes spectra to simulataneously\@
detect and characterize exoplanets.
In this part we have $3$ chapters starting with the introduction to spectral data processing, followed by an introduction
to cross correlation analysis. 
In this chapter we also describe our implementation of cross correlation with the pre-processing.
This is followed by a description of our testbench which is composed of data generation, testing and evaluation.
Finally, we end this part with a retrospective on the limitations on spectral data processing. 

In Part III we discuss the use of our cross correlation algorithm with medium resolution spectra and high contrast images.
This part contains $4$ chapters starting an introduction to high contrast imaging datasets and their dimensionalities.
this is followed by our algorithm to use these dimensions to perform simulataneous detections and characterizations of exoplanets.
