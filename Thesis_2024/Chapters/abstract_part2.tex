\chapter*{Thesis summary}
%Exoplanet detection and characterization have long been treated as disparate and somewhat independent activities when using direct imaging data. 
%In reality characterization of an exoplanet has been conducted as a follow up to a confirmed detection.
%Characterization is typically carried out by comparing spectral features in a simulated exoplanet to the features present \@
%in the data.
%Detection on the other hand, for direct imaging requires the separation of several point like sources in the images as planet\@
%and non planet candidates.
%The non planet point like sources are typically named speckles. 
Traditionally, exoplanet detection and their characterization have been regarded as separate endeavors in the analysis of direct imaging data. 
Detection involves analyzing the images from the direct imaging data and using specific algortithmic strategies to identify those pixels that contain the exoplanet.

To improve this separation a method was suggested to combine spectral and spatial data, in essence combining the detection 
and characterization step.
This has been made possible with the presence of Integral Field Spectrographs (IFS) and the next generation of adapative optics based 
telescopes.
While telescopes have improved and the IFS allows us to extract spectra from every pixel in the telescope's field of view, there is still 
limited post processing that allows us to make subtle distinctions between speckles and exoplanets.
Despite this we have had a plethora of data that has been generated by these instruments, thus data processing itself is complex and requires 
large computing power.
It is in this context that my thesis is situated.

This thesis aims to combine high contrast imaging data with spectroscopic observations and thus create a unified framework to simulataneously,
detect and characterize an exoplanet candidate.
In order to do this we rely on three important pillars,
\begin{enumerate}
    \item Availability of high quality spectra that can be used to discriminate between speckles and exoplanets.
    \item a framework to combine these spectra to high contrast images that does not fully lie in the image or spectral domain alone, and
    \item a set of well tuned post processing algorithms that are correctly benchmarked.
\end{enumerate}
In this regard, my thesis is thus divided into three parts that focus on one of these pillars specifically.
Part II focuses on describing the development of a cross correlation algorithm that processes spectra to simulataneously\@
detect and characterize exoplanets.
In this part we have $3$ chapters starting with the introduction to spectral data processing, followed by an introduction
to cross correlation analysis. 
In this chapter we also describe our implementation of cross correlation with the pre-processing.
This is followed by a description of our testbench which is composed of data generation, testing and evaluation.
Finally, we end this part with a retrospective on the limitations on spectral data processing. 

In Part III we discuss the use of our cross correlation algorithm with medium resolution spectra and high contrast images.
This part contains $4$ chapters starting an introduction to high contrast imaging datasets and their dimensionalities.
this is followed by our algorithm to use these dimensions to perform simulataneous detections and characterizations of exoplanets.
