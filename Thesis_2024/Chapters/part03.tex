\part{Application of cross correlation to high contrast data for simulataneous detection and characterisation}
\startcontents[chapters]
\printmyminitoc{
}



\chapter{Introduction to HCI datasets and their dimensionalities}
Introduction chapter will comprise sections introducing HCI and HRS/MRS. 
It will terminate with explaining why ADI works and also discuss its limitations.
\section{Exoplanet science of high contrast imaging}
Introduce the scientific goals of high contrast imaging
\section{Exoplanet science of medium and high resolution spectra }
Introduce why was medium and /or high resolution spectra necessary
\section{High contrast image data processing}
Introduce ADI and explain the reason it is so successful and also its limitations
\section{Scientific goals of combining high/medium resolution spectra with high contrast imaging}
Introduce the idea of combining them, was there some prior art why or why was it not successful.
\chapter{Data processing for detecting exoplanets in high contrast images and medium resolution spectra}
Use the limitations to explain why the use of spectra was expected to alleviate and improve the scientific case of exoplanet imaging. 
\section{Pre processing of high contrast images to use with spectral data processing}
Describe Valentin's suggestion starting from off center images to centered.
\section{High contrast imaging and spectral data processing}
Describe the use of SDI, ASDI and maybe RDI, finally terminate with explaining molecule maps and the cross correlation mpas
\section{Spectral data processing for MRS and HRS}
Discuss how we developed the spectral data processing and how it could be adapted to HRS data and limitations of both.
SNR computation comes here
\section{Producing detection maps from HCI and MRS data}
Taking off from SNR describe how to apply this to HCI maps and motivate the spatial characterisation
\chapter{Characterization in high contrast images with medium resolution spectra}
Two types of characterization, data and exoplanets.
\section{Characterisation matrix for HCI and MRS data}
Describe why we choose $\rm{T_{eff}}$ and $\log(\rm{g})$ and how they will be identified. 
Limitation is there is no metric to quantify the error bars. Maybe that can already be described here and results in the next section.
\section{Spatial characterizaton of exoplanets}
\section{Use of principal component analysis with high contrast images and spectra}
Describe the use of this for sensitivity characterization

\chapter{Applying our algorithms to datasets}
\section{HD142527}
\section{PDS70}
\chapter{Discussion of the advantages and limitations of processing spectra in this manner}
\lipsum[1]

\chapter{Future steps and how can this be improved}