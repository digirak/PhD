\part{Using cross correlations with high contrast images with several dimensions}
\startcontents[chapters]
\printmyminitoc{
}



%\chapter{Introduction to HCI datasets and their dimensionalities}
\chapter{Producing detection and characterization maps from HCI and MRS data}
\section{Adapting our cross correlation algorithm to images}
\subsection{Pre processing of high contrast images to use with spectral data processing}
Describe Valentin's suggestion starting from off center images to centered.
\subsection{High contrast imaging and spectral data processing}
Describe the use of SDI, ASDI and maybe RDI, finally terminate with explaining molecule maps and the cross correlation mpas
\section{Data processing using this adapted framework}
\subsection{HD142527}
\subsection{PDS70}
\section{Sensitivity analysis of HCI data}
Two types of characterization, data and exoplanets.
Limitation is there is no metric to quantify the error bars. Maybe that can already be described here and results in the next section.
\section{Use of principal component analysis with high contrast images and spectra}
Describe the use of this for sensitivity characterization
\chapter{Using cross correlations with 4D data}
\section{Limitations of map based detection algorithms when using $\rm{SNR}$}
\section{STCM}
\section{Detection sensitivity using map based RoC analysis}
\chapter{Discussion of the advantages and limitations of processing spectra in this manner}
paper discussion + discussion from the other chapters
